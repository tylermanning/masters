\begin{abstract}
The widespread use of computers in everyday living has created a newfound reliance on data systems to support the decisions people make. From wristwatches that monitor your health to fridges that notify users of potential problems, data is constantly being streamed to help users make more informed choices. Because the data has immediate importance to users, techniques that analyse live data quickly and efficiently are necessary. One such group of methods are online change point detection methods. Online change point detection is concerned with identifying statistical change points in a datastream as they occur, as quickly as possible.

The focus for this thesis is on online kernel change point detection methods. Combining kernel two-sample testing and classic change point algorithms, kernel change point methods provide a robust, non-parametric way to measure changes in probability distributions on a variety of datasets and applications. We compare several kernel change point algorithms on several synthetic datasets across a range of measurements that assess online performance. We also provide a novel way to select the kernel bandwidth hyperparameter that adapts to the data in an online fashion.

Additionally, we take a look at the intraday market liquidity changes of several financial markets. We focus on futures instruments of different asset classes from the Chicago Mercantile Exchange.  Data is sampled for the first
four months of 2020 during which the world fell into an economic recession due to a global pandemic. An online kernel change point detection algorithm is applied to detect changes in the market liquidity distribution that are indicative of important macroeconomic events.
\end{abstract}