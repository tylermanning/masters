\chapter{Application to Market Liquidity}

In finance, \textit{market liquidity} describes how quickly a market participant may buy or sell an asset without causing significant fluctuations in the price. In liquid markets, there is minimal impact to the price by quickly buying (selling) it. In illiquid markets, a purchase (sale) of the asset will cause the price to rise (fall) resulting in adverse price selection if the buyer (seller) is making a large transaction. In both cases, market liquidity is in constant fluctuation. It is this fluctuation that is the focus of this thesis. 
By detecting changes in market liquidity a market participant can identify periods of when trading may be more or less favourable. For example, an institutional investor wishing to unwind a large position over the course of several days may want to quickly detect rapid changes in a liquidity that could negatively impact their execution strategy. The following description of the limit order book is based on the notation used in \cite{gould2016queue}. 

\section{Properties of the Limit Order Book}
All modern financial markets are set-up as a double-sided auction between buyers (bid side) and the sellers (ask side) called limit order books (LOBs). A market participant may post orders at specific prices on either side of the book. A limit order, denoted by $x$, is defined as a tuple containing a price, $p_x$, and a size, $w_x$ where $|w_x|>0$:

\begin{equation}
x = (p_x, w_x).
\end{equation}

A limit order's size indicates how many units of an asset a buyer (seller) is willing to trade at $p_x$. The prices for which an order can be submitted at are discrete prices. 

The LOB at time, $t$, is denoted as $\mathcal{L}(t)$, and it represents the set of all orders currently active at time $t$. All bid orders are at the best bid price or lower. The best bid price at time, $t$,  is denoted as:
\begin{equation}
b(t) =  \max_{\{x \in \mathcal{L}(t)|w_x<0 \}} p_x
\end{equation}

All ask orders are placed at the best ask price or higher. The best ask price at time, $t$, is denoted as:
\begin{equation}
a(t) = \min_{\{x \in \mathcal{L}(t)|w_x>0 \}} p_x
\end{equation}

The \textit{bid-ask spread} or simply the spread at a time, $t$, is defined as:
\begin{equation}
s(t) = a(t) - b(t)
\end{equation}

The normalized spread, $s_n(t) \geq 0$, can be defined by: 
\begin{equation}
s_n(t) = \frac{s(t)}{\pi} - 1
\end{equation}

There are many types of orders that can be placed in the LOB. In this context, we assume all orders in the LOB at a given time are basic limit orders. That is, they are orders placed at a specific price with a specific size. Therefore, at a price, $p$, and a time, $t$, the total number of orders is:
\begin{equation}
n^b(p, t) = \sum_{\{x \in \mathcal{L}(t)|w_x<0 , p_x=p\}} |w_x|
\end{equation}

On the ask side, it is similarly defined as:
\begin{equation}
n^a(p, t) = \sum_{\{x \in \mathcal{L}(t)|w_x>0 , p_x=p\}}  w_x
\end{equation}

Everytime a trade occurs or a resting limit order is cancelled at that price, size is removed from the LOB at that price. If the size at the $b(t)$ reaches zero then the spread increases the $b(t)$ decreases. 

Every LOB has two configuration parameters: the tick size, $\pi >0$, which is the smallest possible price increment between orders at different prices and lot size, which dictates the smallest amount that can be traded. As mentioned in \cite{gould2016queue}, LOB is essentially a one-dimensional lattice where the dimension is price and every point on the price axis is a multiple \footnote{In most cases these multiples of $\pi$ are strictly positive but in particular markets, such as ED packs and bundles, there may be negative integers of $\pi$, i.e. a price axis with positive and negative prices.} of $\pi$.

There are two possible distributional changes that are of interest. The first is how the entire distribution of liquidity changes over time. These changes are often associated with events happening in the market such as the start of U.S. trading hours, the release of economic data or a planned speech by the head of the federal reserve. There may also be unplanned events that affect the overall liquidity in the book such as a the 9/11 terrorist attack or a virus outbreak like the Corona virus.

The second is how the distribution of the bid orders differs from the distribution of the ask orders. Typically, the distribution of liquidity on each side of the book is more or less symmetric. However, there have been studies that have shown times when there is an asymmetry between the two sides. This asymmetry has been shown to correlate to price changes. 



\section{Dataset Construction}

While many types of markets operate using a basic order book, this thesis will be focused on futures markets found in the Chicago Mercantile Exchange (CME). The reasoning for this is two-fold. The first, unlike bond markets or currency pair markets, liquidity for many futures products is concentrated on a single exchange. This simplifies analysis of liquidity by not having to aggregate data across several exchanges into a synthetic ladder. The second advantage is, unlike equity markets where stocks and exchange traded funds (ETFs) are traded, there are no dark pools available for the futures markets under consideration in this thesis. Again this simplifies where the liquidity data for an instrument may appear.

The futures that will be used in our dataset are based on a combination of selecting the most traded contracts on the CME and selecting futures that represent different asset classes. They are summarized in the following table:

\begin{center}
\captionof{table}{Summary of Futures Studied}
\begin{tabular}{SSSSSS} \toprule
    {Name} & {Asset Class} & {CME Symbol} & {Tick Size} & {Other}  \\ \midrule
    {E-mini S\&P 500}  & {Equity} & {ES} & 0.25 & 1.402  \\
    {10 Year T-Notes}  & {Fixed Income}  & {ZN} & 3.442  & 0.299  \\
    {Crude Oil}  & {Commodity}  & {CL} & 3.442  & 0.299  \\
    {Gold} & {Commodity}  & {GC} & 3.442  & 0.299  \\
    {Euro FX}  & {Currency}  & {6E} & 3.442  & 0.299  \\ \bottomrule
\label{futures}
\end{tabular}
\end{center}

For each future contract in table \ref{futures}, the state of the order book is sampled every \hl{60 seconds} from 7 AM to 4 PM EST for a period of \hl{45} days. The characteristics of the order book that are sampled at every time step, $t_I$, are:
1) Quantity, $q$, at the top $N$ levels of the book, $q_1 ... q_N$ 
2) Spread, $s: [0, \mathbb{Z}^+]$, normalized by tick size, where spread of zero indicates there is no spread.

While spread is not indicative of liquidity but rather an absence of liquidity, we believe it is an important feature to include because all other things being equal, a change in spread is an important change in liquidity distribution for market participants. 