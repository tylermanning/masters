\chapter{Kernel Changepoint Detection}

\section{Related Work}

One of the first papers to use the term kernel change-point detection was in \cite{desobry2005online}. The authors present an on-line kernel change point detection model based on single class support vector machines ($\nu$-SVMs). They train a single class support vector on a past set, $\mathbf{x}_{t,1}={x_{t-m_1},...,x_{t-1}}$ of size $m_1$ and train another single class support vector on a future set $\mathbf{x}_{t,2}={x_t,...,x_{t+m_2-1}}$ of size $m_2$. A ratio is then computed between the two sets that acts as the dissimilarity measure in Hilbert space. If the points are sufficiently dissimilar over some predetermined threshold,$\eta$, then a change point is assigned to the time spitting the two sets of data. The authors argue that a dissimilarity measure between kernel projection of points in a Hilbert space should estimate the \textit{density supports} rather than estimate the probability distributions of each set of points. While this approach inspired a lot of interesting research that will be discussed below, modern approaches have proven to be more practical.

In 2007, Harchoui and Cappe \cite{harchaoui2007retrospective} approached the off-line change point problem with a fixed number of change points by using kernel change point detection. This was further extended to an unknown number of change points in 2012 by Arlot et al. \cite{arlot2012kernel}. Finally, Garreau and Arlot extended this in line of research kernel change points in the off-line setting of detecting change points. Fundamentally, their method is the kernel version of the following least squares optimization problem:

\begin{equation}
J(\tau, \mathbf{y}) = \frac{1}{n} \sum_{k=1}^K(\tau) \sum (Y_i - \hat(Y_k)^2 + \beta \text{pen}(\tau)
\end{equation}


The benefits of this off-line kernel change point detection is that it operates on any kind of data for which a kernel that properly reproduces a Hilbert space can be applied. For example, it can be applied to image data, histogram data, as well as $d-$dimensional vectors in $\mathbb{R}^d$. Garreau shows their KCP procedure outputs an off-line segmentation near optimal with high probability. Lastly, the authors recommend choosing the kernel based on best possible signal to noise ratio that the distribution gives based on $ \Delta^2 / M^2$ . Therefore, some prior knowledge or training set is necessary for calibrating the kernel. 

In  \cite{li2015m}, the authors make use of the B-test introduced in \cite{zaremba2013b} and develop an offline and online change-point detection algorithm called the MStats algorithm. The authors also refer to this algorithm as the Scan-B algorithm in a follow-up paper. At each time-step, the online model samples new data from a window of size $B_0$ and does a B-test with $N$ past samples that are kept as reference samples. 

$$Z_{B_{0}, t}:=\frac{1}{N} \sum_{i=1}^{N} \operatorname{MMD}_{u}^{2}\left(X_{i}^{\left(B_{0}, t\right)}, Y^{\left(B_{0}, t\right)}\right)$$

The resulting test statistic is then normalized by $Z_{B_{0}, t}/\sqrt{Var[Z_{B_0}]}$ where the authors provide a theoretical calculation of $\text{Var}[Z_B]$. If the normalized test statistic exceeds some predefined threshold then a change-point is declared. The B-test is memoryless in the sense that the statistic is calculated each time and only the value of the last calculation has any weight. This is similar to a control chart that calculates a z-score at each iteration. Adjusting the size of the window, $B_0$, results in the usual trade-off of performance in online change-point detection. A smaller block size will have a smaller computational cost and a smaller detection delay but will result in a worse test power resulting in higher type II error.

From here, theoretical bounds are developed for the average run length and expected detection delay. Experiments are done on real-data sets including a speech dataset  and the Human Activity Sensing Consortium (HASC) dataset where the performance was better than the relative density-ratio (RDR) algorithm described in \cite{liu2013change}.

More recently in \cite{chang2019kernel}, a  kernel change-point detection method is proposed that uses deep generative models to augment the test power of the kernel two sample test statistic. They point out MMD's lack of test power when using limited samples from the new distribution, $Q$, which may easily leading to over-fitting with kernels. Thus they use a generative adversarial neural network (GAN), trained on historical samples of $X \sim  P$  with noise injected into $X$. This surrogate distribution is then used in conjunction with possible change-points to improve the test power of a modified MMD measure that makes use of compositional kernels.

The method is compared to other prominent change-point methods for off-line  change detection such as the aforementioned MStats-KCPD, LSTNet, and Gaussian process change-point models. All comparisons are done on synthetic data with piece-wise i.i.d. data. All methods are benchmarked using the AUC metric for classification performance and it is shown the KL-CPD method is competitive or better than state of the art methods.  Furthermore, the AUC performance is maintained as the dimensionality of the data is increased, making their kernel learning framework very interesting for future off-line change-point detection. It remains to be seen if this framework can be adopted in an on-line context where time to detection is a key constraint on practicality.

In the on-line setting, several methods use kernel embeddings with a two-sample hypothesis test. This is done in a similar vein to the classic CUSUM and Shewart control charts. They all make use of the maximum mean discrepancy (MMD) test statistic for a two-sample kernel hypothesis test. 

%https://media.nips.cc/nipsbooks/nipspapers/paper_files/nips28/reviews/1852.html

A modified, "no-prior-knowledge" exponentially-weighted moving average (NEWMA) is introduced in \cite{keriven2018newma}. Based on the standard exponentially weighted moving average, NEWMA computes two EWMA statistics of different weights. If the difference between the two EWMA statistics exceeds a predefined threshold then a changepoint is declared at that time step. The point of using two EWMA statistics is for one to have a larger forgetting factor. This makes any recent changes in distribution to weigh heavily in one statistic, resulting in a large difference between the two statistics. 

Since a standard EWMA is a parametric method, the authors apply a kernel mapping function, $\Psi$, to the data prior to applying the exponential factors. This provides a memoryless, non-parametric, online changepoint detection method that does not need to constantly store all previously streamed data. Once the statistics are updated at each iteration, the raw data may be discarded. 

While kernel mean embeddings could be used for approximating, $\Psi$, as is the case for standard implementations of MMD, this would require the storage of past examples of data. Because the authors aim to reduce run-time cost and storage cost, they use a Random Fourier Features (RFF) approach for the mapping $\Psi$. (Note RFF is sometimes referred to as \textit{random kitchen sinks}) There are several approaches available for optimizing the RFF approach that are well studied in the literature. The authors make use of the standard RFF implementation, the FastFood implementation introduced in \cite{le2014fastfood}, and Optical Processing Unit implementation from \cite{saade2016random}. 

Experiments are done with the three implementations of RFF and are compared to the MStats (Scan-B) algorithm using a Gaussian kernel. Experiments with synthetic datasets are run using streaming data that is generated from different Gaussian mixture models. They also use an audio dataset for testing on real data. The variants of NEWMA are very similar, if not better than MStats (Scan-B) in terms of missed detection percentage. In terms of the detection delay and false alarm trade-off, the NEWMA algorithm and its variants appear to be mildly better as well. The largest advantage of the NEWMA variants over the MStats (Scan-B) method is in the execution time. Because MStats (Scan-B)' s execution scales linearly with window size, while NEWMA's execution time does not depend on window size.

Finally, in a recent, preprint paper \cite{flynn2019change}, a kernel CUSUM (KCUSUM) algorithm is proposed, where the classic CUSUM algorithm is adapted using the MMD statistic for on-line detection. The authors use a modified, unbiased MMD statistic that can be computed in linear time. This formulation of the MMD statistic was originally defined in section 6 of \cite{gretton2012kernel} as:

$$\operatorname{MMD}_{l}^{2}[\mathcal{F}, X, Y] :=\frac{1}{m_{2}} \sum_{i=1}^{m_{2}} h\left(\left(x_{2 i-1}, y_{2 i-1}\right),\left(x_{2 i}, y_{2 i}\right)\right)$$

Where,

$$h\left((x_i, x_j), (y_i, y_j)\right):=k\left(x_{i}, x_{j}\right)+k\left(y_{i}, y_{j}\right)-k\left(x_{i}, y_{j}\right)-k\left(x_{j}, y_{i}\right)$$

The algorithm functions as follows, every two observations, the MMD$_l$ is calculated using newly observed data points and data points sampled from some \textit{reference} distribution that is known at the outset. This reference distribution can be thought of as the "in-control" distribution of the data-stream that new observations are compared to. The calculated MMD$_l$ acts as the update term to the cumulative sum statistic, hence the name KCUSUM. If this kernel cumulative sum statistic exceeds some predefined threshold, then a change-point is identified. 

Besides its speed of computation, an additional benefit of MMD$_l$ is it is normally distributed under the null distribution unlike the quadratically-calculated MMD. This facilitates analysis of bounds and provides statistical guarantees for worst-case detection delays and time to false alarm rate. While this non-parametric approach can detect any change in the distribution of a sequence, it does struggle with more complicated distributional changes such as variance changes of a single dimension and changes beyond first and second-order moments.

%In 2016, James et al. \cite{james2016leveraging} proposed using energy statistics to test the significance of a change point that is robust to anomalies. They point out that their work is the first to accurately detect change points with fast time to detection while not being affected by extreme outliers that are not change points. They compare their E-divisive with medians algorithm to the parametric PELT technique. 

\section{Our Approach}

This section describes our novel method for change-point detection.