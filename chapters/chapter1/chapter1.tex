\chapter{Introduction}
\section{Motivation}
Recent advances in neural network architecture and utilization of large amounts of data have led to astonishing results in fields such as image recognition, natural language processing, and speech recognition [add citations]. Many modern techniques such as long short term memory neural networks and convolution neural networks have been applied to time series forecasting and, while the results are promising, it remains to be seen whether such techniques can significantly outperform classical statistical methods \cite{makridakis2018statistical}. This opens the door for novel ideas to be tried and benchmarked against existing, classical methods.

In this thesis, we concern ourselves with abrupt regime changes in time series that indicate a change in the underlying distribution of a time series. There are several constraints when tackling this problem that need to taken into consideration when choosing a model. 

The first is the nature of the data for the particular use case that the model will be used in. If the data is being generated by processes that have no fundamental or well-understood process then modelling particular probability distributions becomes intractable and does not apply. Therefore, models that do not model particular probability distributions or even assume that there is one to model must be used. Methods that fall into this category are known as \textit{non-parametric} models and are more flexible for data that do not easily fall into a well known structure.% Usually this added flexibility is a trade-off with performance and/or speed with parametric models.

The second consideration is to determine what in what scenario the data is analysed. Some algorithms are off-line or batch algorithms in that they are applied in an ex post fashion after the dataset has been completely acquired. The first attempts are change point detection using control charts by Shewart were done in an on-line fashion as the data is streamed in a near, real-time fashion. In the literature, on-line methods of change point detection are also sometimes called sequential change point detection. For this thesis, we will use the terms interchangeably.  %It is possible to convert an on-line algorithm to an off-line algorithm by simply 

The third consideration that is a consequence of detecting change points in an on-line setting is the presence of outliers. 

The fourth consideration is determining if there are multiple change points or to assume there is only a single change point to detect. 

\section{Previous Work}


In 2016, James et al. \cite{james2016leveraging} proposed using energy statistics to test the significance of a change point that is robust to anomalies. They point out that their work is the first to accurately detect change points with fast time to detection while not being affected by extreme outliers that are not change points. They compare their E-divisive with medians algorithm to the parametric PELT technique. 

\section{Problem Formulation}
In the 2017 survey done by Aminikhanghahi and Cook \cite{aminikhanghahi2017survey}, the authors give the following definition of a \textit{change point}:
A change point represents a transition between different states in a process that generates the time series data.

Generally speaking, the fundamental change point problem is the following:
\begin{itemize}
  \item $H_0: P_{X_m }=  P_{X_m+k} $
  \item $H_A: P_{X_m} \neq  P_{X_m+k} $
\end{itemize}

\section{Data Sources}

Inspired by twitter trend detection research, the ultimate goal is to test our contribution on labelled twitter count data that have some mix of trending and non-trending sequences. Since this data would be costly to acquire and label. This will not be used to test the performance of this work. Instead, the following sources of data will be used:
 \subsection{Synthetic datasets}
\subsection{Private dataset}
  
\section{Review of 2017 Garreau PhD }
This PhD is an extension on the 2012 paper by Arlot et al. \cite{arlot2012kernel} that proposes a type of kernel change point detection. Their non-parametric method is concerned with the off-line setting of detecting change points. They tackle the possibility of multiple change points.


\section{Our Contributions}
Lay out the contributions of this work.