\chapter{Introduction}
\section{Motivation}
Recent advances in neural network architecture and  utilization of large amounts of data have led to astonishing results in fields such as image recognition, natural language processing, and speech recognition [add citations]. 

Unfortunately, it remains to be seen whether such techniques can significantly outperform classical statistical methods \cite{makridakis2018statistical}. This opens the door for novel ideas to be tried and benchmarked against existing methods. 

Time series classification in general has seen extremely competitive results using nearest neighbour techniques \cite{bagnall2017great}. There are many variations on how to do this but fundamentally it involves a distance calculation between two sequences, followed by a one nearest neighbour classifier. The most common distance functions used for this is the Euclidean distance or some variant thereof. 


Detecting abrupt changes in a time series is a common task in time series analysis and signal processing. Many outlier detection methods exist such as . A closely related topic is change point detection which focuses on discerning  when possible regime shifts occur in a time series. Here the regime shifts indicate collective outliers and are considered out of place when compared to another regime in the time series.

\section{Previous Work}
Broadly speaking there are two main categories that methods fall into, parametric and non-parametric modelling. 

Nikolov et al. describe a latent source model that is a data-driven non-parametric classifier that detects trends on twitter using the counts of a twitter string or hash-tag. 

\section{Our Contributions}
Lay out the contributions of this work.