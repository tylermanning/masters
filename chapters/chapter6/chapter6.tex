\chapter{Conclusion}
\section{Summary of Thesis}
In chapter 1 we briefly introduced the change point detection problem and different considerations one must make when trying to solve it. Some high level applications are presented to motivate the topic of change point detection especially in finance.

Chapter 2 dives deeper into the online change point detection and its origins from hypothesis testing. A formal description of the online detection problem is presented along with common evaluation metrics. Fundamental online change point algorithms such as Shewart control charts, CUSUM and exponentially weighted moving averages are covered as well. 

Chapter 3 covers kernel change point detection and all the kernel machinery necessary for understanding the kernel-based statistic used in online change point detection. A review of the most recent kernel change point detection literature is covered and three recent methods are explained in detail as they are used for experiments in chapter 4. A novel method for an online calculation of the median heuristic is also presented. 

Chapter 4 compares the kernel versions of EWMA, CUSUM and Shewart control charts using various synthetic datasets. The results across various performance metrics are discussed and the particular use cases for using a rolling median heuristic is presented as well.

Finally, in chapter 5 we apply a kernel change point method to market liquidity data for various future contracts on the Chicago Mercantile Exchange. A brief overview of the limit order book and how liquidity fluctuates is presented as well.  

The novel aspects of this thesis include the wide array of experiments run across various online kernel change point methods. The introduction of an online median heurtistic calculation that can fit into any online change point detection method that relies on the median heuristic for kernel bandwidth tuning. Finally, we explore market liquidity of several financial future instruments on the CME by running a non-parametric online algorithm on a new market liquidity dataset.

\section{Discussion and Future Work}

There are several areas for which this thesis could be immediately improved upon. The first is further optimizing the bandwidth selection for the Gaussian kernel. While we presented a fast, online method of updating the median heuristic for setting this bandwidth, several questions still remain. How often should this bandwidth be updated? We went with an algorithm that updated the bandwidth every hundred timesteps because that seemed to keep the median estimate robust to changes in the data. However, experimentally we did notice that for small windows the median was very volatile, meaning there is a certain optimal window size that needs to be selected for this online bandwidth update. Perhaps developing the theory around this median update would help provide minimum window size guarantees that practitioners can implement.

Secondly, while we tried to cover many different type of experiments in chapter 4, the goal was to highlight the difficulty in exhaustively testing change points non-parametrically in an online fashion. The hope of this is to spur more research in this direction so that future methods may all be compared against a holistic dataset that is agreed upon by the research community. Furthermore, any future online change point algorithms should be compared against some variant of the NEWMA algorithm for benchmarking purposes. It is a very lightweight, fast algorithm that performs well in a variety of circumstances. Therefore, researching any improvements of NEWMA would be worth the effort as well. 

Another focus of this thesis was using the MMD for online change point detection. It would be interesting to explore other  distance measures that could compare two distributions non-parametrically. It would be interesting to re-use the same approaches explored in this thesis, but with the MMD swapped for a different distance measure. One example that comes to mind is the kernelized Stein discrepancy introduced in \cite{liu2016kernelized} for goodness-of-fit tests. By framing the change point detection problem as comparing new, incoming data to some stable signal, this can be essentially turned into a goodness-of-fit test where one tests how well the new data fits to the signal over time. Other possible ways to compare two distributions are divergences measures that are a well studied subject in many domains \cite{basseville2013divergence} \cite{pardo2018statistical}. Finally, several techniques using estimations of probability densities have been explored for online change point detection. Because these techniques estimate densities directly from the observed data, they have promising results in the non-parametric setting \cite{kawahara2012sequential} \cite{bouchikhi2019kernel}.

Recently, interesting research has been made on re-purposing binary classifiers as two-sample tests. In \cite{hediger2019use}, random forests are compared to different implementations of MMD. The authors run many tests comparing the test power across the different two-sample tests and their random forest. The results are interesting because while the random forest classifier is not better in every situation, it is better on hard two-sample tests such as the blobs dataset. Therefore, it remains to be seen whether these classifier approaches to two-sample testing can be adapted to online change point detection in an efficient manner. Framing hypothesis testing as binary classification is also explored in \cite{lopezrevisiting}. This may help bridge the statistics research in change point detection that is more theoretical based with more empirical based results from the machine learning community. 

On the finance side, we believe there is a lot of potential for researching the mechanics of market liquidity. The fluctuations in liquidity in worldwide markets have massive economic implications. So much so that financial firms that provide liquidity were deemed an essential service during the 2020 covid-19 lockdown and allowed to remain open as most other businesses were forced to temporarily shutdown. This means studying the implications of market liquidity changes is vital to maintaining a robust modern economy. However, as far as we can tell there are almost no significant datasets for market liquidity publicly available.  Hopefully, in the future more companies who own this type of data will release it periodically to the research community when possible. This would encourage researchers to tackle more interesting problems within real datasets. For example, one problem that exists when examining real datasets like market data is the inexistence of ground truth change points. This makes it incredibly hard to evaluate any proposed method for detecting a change point and disincentivizes research because it is so difficult to concretely say if a change point method is doing anything useful. In this thesis, we used our knowledge of particular products and macroeconomic events for assessing performance but we'd like to see if more robust ways are possible. Perhaps comparing the results of an online algorithm like NEWMA to an offline method will reveal if the segmentation is similar between the two. Thereby, providing conclusive results about an online change point technique.